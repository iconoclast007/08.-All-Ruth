\section{Ruth 1 Outlines}


\subsection{My Outlines}

\subsubsection{Coming Back Broken}
\index[speaker]{Keith Anthony!Ruth 01 (Coming Back Broken)}
\index[series]{Ruth (Keith Anthony)!Ruth 01 (Coming Back Broken)}
\index[date]{2017/03/21!Ruth 01 (Coming Back Broken) (Keith Anthony)}
\textbf{Introduction: }Naomi's return to Bethlehem pictures Israel return back to the land during the church Age. She comes back bitter and broken, but gets blessed by her connection to Ruth and Boaz. Later she gets to witness a wedding between the two!
\begin{compactenum}[I.][7]
    \item The \textbf{Longing} (for food) \index[scripture]{Ruth!Ruth 01:01}(Ruth 1:1)
    \item The \textbf{Looking} (at Moab) \index[scripture]{Ruth!Ruth 01:01}(Ruth 1:1)
    \item The \textbf{Leaving} (the land of faith for the land of the flesh) \index[scripture]{Ruth!Ruth 01:01}(Ruth 1:1)
    \item The \textbf{Loosing} (husband and sons) \index[scripture]{Ruth!Ruth 01:03}\index[scripture]{Ruth!Ruth 01:05}(Ruth 1:3, 5)
    \item The \textbf{Learning} (why did she have to suffer loss to know she was in the wrong place?) 
    \item The \textbf{Listening} (why was this a surprise to her?)  \index[scripture]{Ruth!Ruth 01:06}(Ruth 1:6)
    \item Ruth's \textbf{Longing} (Naomi was oblivious to Ruth's faith)  %\index[scripture]{Ruth!Ruth 01:06}(Ruth 1:6)
\end{compactenum}


\subsection{Outlines from Others}

\subsubsection{Three Tombstones in a Washpot}
\index[speaker]{unknown!Ruth 01:01-07 (Three Tombstones in a Washpot)}
\index[series]{Ruth (unknown)!Ruth 01:01-07 (Three Tombstones in a Washpot)}
\index[date]{unknown!Ruth 01:01-07 (Three Tombstones in a Washpot) (unknown)}
\textbf{Source: }sermonnotebook.org\\
\textbf{Introduction: }The little book of Ruth has been called ``the greatest piece of literature ever written.'' Another writer called the story of Ruth ``the Cinderella of the Bible.'' It is the story of how a pagan girl named Ruth came to be part of the covenant people of Israel. In the 100 verses that make up the book of Ruth, we see this young woman as she is pursued by grace, and brought out of her wretched condition. This is a story of redemption, of love, of grace and of hope. It is a story we need to become familiar with on a very intimate level.  n these first seven verses of Ruth we are introduced to the family of a man named Elimelech who lived during the days of the judges, v. 1. It is the sad tale of a man who chooses to walk out on the Lord and on God's plan for his life. As a result of his decision, he and his family pay a terribly high price. We are told that Elimelech takes his family to a place called Moab. Moab was located just across the Jordan River, east of the Promised Land. It was inhabited by people who worshiped pagan gods. The Moabites were the descendants of a man named Moab who was the son of an incestuous relationship between Lot and on of his daughters, Gen. 19:30-38. They were a proud people noted for their lawlessness, immorality and brutal violence, Lev. 18:24--25; Deut. 9:4--5; Isa. 16:6; Psa. 60:8. They attacked and opposed Israel, seeking to destroy the people of God, during Israel's wilderness wanderings, Num. 23--25; Deut. 23:3---6. This was a people opposed to God and His ways. In Psalm 60:8, God says this, ``Moab is my washpot $\hdots$ '' This phrase means that they were a despised thing, compared to a vessel containing water to be used by slaved to wash the feet of a conquering hero. God says that they are nothing and that they will be reduced to the lowest form of slavery! Yet, they were a people who could have been saved had they repented of their sins as Ruth did. It is to this despised and wicked nation that Elimelech moves his family. Here we see a picture of that person who willingly turns his back on the things of God and pays an awful price. If this section of scripture teaches us anything, it teaches us that living in a backslidden condition carries with it devastating consequences, but repentance and restoration are always a possibility. With this information in mind today, let's look into this passage of scripture today as I preach on the thought Three Tombstones In A Washpot.
\begin{compactenum}[I.][7]
    \item A Time of \textbf{Desperate Circumstances} \index[scripture]{Ruth!Ruth 01:01}(Ruth 1:1)
    \begin{compactenum}[A.][7]
    	\item A \textbf{Material Famine} in the Land -- Verse 1 describes the situation that Elimelech and his family faced by telling us that there was ``a famine in the land.'' A famine is an extended dearth. A time when food is in serious shortage. While there was a shortage of foods in the land at that time, it wasn't the only famine the people of Israel faced.
    	\item A \textbf{Moral Famine} in the Land -- Verse 1 tells us that this story takes place during the time of the Judges. The attitude of the people during those days is summed up in the last verse of the book of Judges, Judges 21:25. It can best be described as a time of turbulence and social upheaval. They were days marked by lawlessness, idolatry, false religion, theft, drunkenness, homosexuality, sexual perversion, violence, national division, civil war and extreme unbelief. Days, it would see, that are not too different from the days in which we find ourselves. Of course, when man ``does that which is right in his own eyes'', what else should we expect?
    	\item A \textbf{Missionary Famine} in the Land  -- 	Often, in the Old Testament period, God used famine as a tool of discipline. When His people strayed away from Him, He would reach out to them to call them back to Himself by orchestrating a famine, Deut. 11:16-17; ; 2 Chron. 7:13-14.
    \end{compactenum}
	\item A Time of \textbf{Dangerous Choices} \index[scripture]{Ruth!Ruth 01:02}(Ruth 1:2)
    \begin{compactenum}[A.][7]
    	\item Choose to \textbf{Leave the Promised Land} -- Verse 1 tells us that this man made a conscious decision to leave ``Bethlehem-Judah'' for the country of Moab. The name ``Bethlehem'' means ``House of Bread'' and the name ``Judah'' means ``Praise''. At the present time neither of those places were living up to their names. There was no bread in "The House Of Bread" and there was no reason for rejoicing in "Praise". However, while the geographic locations failed to live up to their names, so did Elimelech! His name means "My God Is King." If that had been true about this man, he would have known that God's valleys do not last forever and that God would have taken care of His people. After all, Elimelech had a close relative name Boaz. He stayed in Bethlehem and seemed to do quite well in spite of the famine, Ruth 2:1. Yet, Elimelech chose to leave his inheritance in the promised land and head off to a land where God would not bless him!
    	\item Choose to \textbf{Live in a Polluted Land} -- To leave Israel to go to Moab was to violate the clear commandment of the Lord, Josh. 23:7, 12. Yet, Elimelech chose the forbidden path to Moab over contentment in the things of the Lord. To leave one's inheritance as Elimelech did was equivalent to denying the faith of Jehovah. It was a total turning from God to the world. You see, for Elimelech and his family, this move to Moab involved total separation from the things of God. They could not worship at the Temple, they could not bring their offerings, they could keep the feasts that were commanded by the Law. They were totally isolated from everything that stood for God. Not only that, by moving his family to Moab, Elimelech exposed his family to evils they would have avoided had they stayed in Israel. For instance, both of those boys married pagan women. It is never right for a child of God to marry an unbeliever, 2 Cor. 6:14. These men violated the will of God by intermingling with this pagan race, Ezra 9:1-2; Neh. 13:23.
    	\item Choose to \textbf{Linger in a the Prodigal Land} -- This family went to Moab and there they stayed! The word "continued" means "to exist or to become." This may indicate that not only did Elimelech and his family go into Moab, but that Moab got into them as well. Who knows how far into the sin and society of Moab this family fell. I am sure when they left they told themselves that it would only last a short time. But the days turned into weeks, the weeks into months and the months into years. Before they knew it, 10 years had passed, v. 4, and they were farther away from the Lord than could have ever imagined themselves being.
    \end{compactenum}
     \item A Time of \textbf{Deadly Consequences} \index[scripture]{Ruth!Ruth 01:03--05}(Ruth 1:3--5)
    \begin{compactenum}[A.][7]
    	\item There was \textbf{Discipline} in that Home -- Before Elimelech loaded up his family to move to Moab, it is likely that he had already moved away from God in his heart. You see, no one just wakes up one morning and decides to leave God behind! It is a slow subtle process that builds until the believer has moved farther away from the Lord than he could have ever anticipated. The reason I say there was discipline in that home lies in the names of Elimelech's sons. There names were Mahlon which means "Sick" and Chilion which means "Pining, or Wasting Away." It may be that these two boys were just the victims of circumstance. But, if you believe that, then you don't believe in a God of providence! I think God was trying to get Elimelech's attention long before he ever left for Moab.
     	\item There was \textbf{Death} in that Home -- In spite of his relationship to the Lord, in spite of the blessings of God, in spite of the intervention of the Lord in his family, Elimelech moved to Moab. What was a reality in his heart became a reality in his life! After he was there for some time, he died. After a while, his sons were also taken in death. It is my opinion that God used the ultimate form of discipline in the lives of Elimelech and his family in order to bring the remainder of the family back to God. You see, the famine in Bethlehem was a call to repent. Those sick sons were a call to repent. God gave this family ten years of rope, but time finally ran out and they paid the ultimate price for their disobedience.
    	\item There was \textbf{Defeat} in that Home -- Naomi and her two daughters-in-law were left desolate. In that society, the poorest of the poor were widows with no children to care for them. These women were left with nothing but desolation, discouragement and defeat. It didn't have to be, but because of the sin in their hearts and the choices they made in life, they were forced to pay a horrible price. All Naomi had to show for her disobedience and backsliding were three tombstones in the land of Moab. All she had left were her daughters-in-law and she couldn't even provide for them. These women were in a desperate condition.
   \end{compactenum}
        \item A Time of \textbf{Deliberate Changes} \index[scripture]{Ruth!Ruth 01:03--05}(Ruth 1:3--5)
    \begin{compactenum}[A.][7]
    	\item a Time of \textbf{Realization}  -- Somehow or another Naomi heard that the Lord was again blessing His people. In truth, He had never stopped blessing! Did you know His chastisement is also His blessing? If he didn't love you He would chastise you when you failed. Anyway, someone came into Moab with the good news that God was blessing back in Israel. This sparked a desire in her heart to go home. Maybe she remembered what it was like to be close to the things of God. Maybe she remembered the sacrifices and the worship. Maybe she missed the sweet fellowship she had enjoyed with the people of God. Whatever the thoughts were that went through her head, she finally woke up in Moab and wanted to go home. This is what happened to the Prodigal Son, Luke 15:17, "And when he came to himself..." (Ill. Being in a Backslidden condition is a form of insanity.) When that boy saw where he was and what he was missing out on, he wanted to go home. So it was with Naomi.
   		\item a Time of \textbf{Repentance}  -- The Bible says that she "Arose...that she might return from the country of Moab", v. 6; and "Wherefore she went out of the place where she was..." Naomi rose up and left Moab behind! She experienced a change of heart that resulted in a change of action. She is a picture of someone repenting of their time in Moab and returning home.
   		\item a Time of \textbf{Returning}  -- Naomi rose up and went on her way to return to the land of Judah. She was headed back to the land of "Praise". She was going back to where she should have been all along! She was going home!
   \end{compactenum}
\end{compactenum}
\textbf{Conclusion: }When Naomi went into Moab, by her own testimony, she went in full, Ruth 1:21. When she came out, she came out empty. All of her hopes, all of her dreams and all of her to morrows were reduced to three tombstones in a washpot called Moab. When she left that country she left everything she valued behind. When you go to Moab, you always leave something! Naomi left three tombstones in her Moab, what will you leave behind? Will you leave you testimony? Will you leave you innocence? Will you leave your health? Will you leave your wealth? Will you leave you family? What will your journey away from God cost you? Whatever it is, it will cost less to come home today than it will if you wait to a later date. You will leave less behind if you come now than if you wait until later. God can salvage more from your life now than He can if you linger in that polluted country. By the way lost friend, this message is for you too! God is calling you to leave your wretched life of sin behind and to come to Him and be saved. If you linger there away from Him, you will pay the ultimate price. You will lose your soul in Hell! It doesn't have to be that way. If the Lord is calling you to be saved, get to Him today and let Him take care of that for you!


\subsubsection{Ruth's Journey}
\index[speaker]{unknown!Ruth 01 (Ruth's Journey -- Example of a Steadfast Life)}
\index[series]{Ruth (unknown)!Ruth 01 (Ruth's Journey -- Example of a Steadfast Life)}
\index[date]{unknown!Ruth 01 (Ruth's Journey -- Example of a Steadfast Life) (unknown)}
\textbf{Source: }sermonnotebook.org\\
\textbf{Introduction: }There are many stories in the Bible that serve as an encouragement to our hearts. The story of Ruth is no exception. Many look at this little book, which was written during the time if the judges, and see nothing more than a love story. However, while there is a love story of sorts in this book, that is the most shallow interpretation. The bigger picture is that of a lost sinner who, through divine guidance and providence, is brought into a relationship with Jehovah and is made to be an ancestor of the Lord Jesus Christ. We are taught in this book that God is not just the Savior of Israel, but that He is the Savior of the entire human race!
I wish there was time this evening to preach this entire book tonight, but there isn't. What I would like to do is to look into the verses we have read this evening and pull out a little snapshot of the heart of this woman named Ruth. In these verses, and throughout this book as well, we see in Ruth a tremendous example of a steadfast life. She teaches us about remaining faithful even when others around us do not. As we have time this evening, let's look into these verses and examine this Example Of A Steadfast Life.%\footnote{From http://www.sermonnotebook.org/old testament/Ruth 1_1-22.htm}
\begin{compactenum}[I.][7]
    \item Ruth's \textbf{Condition} \index[scripture]{Ruth!Ruth 01:01--07}(Ruth 1:1--7)
    \begin{compactenum}[A.][7]
    	\item Her Ancestry \index[scripture]{Ruth!Ruth 01:04}(Ruth 1:4) -- Ruth was a member of a condemned nation. She and her people were sinful and had been judged and condemned by the Lord. Her's was a desperate and lost condition.
    	\item her Adversity \index[scripture]{Ruth!Ruth 01:05}(Ruth 1:5) -- This verse tells us that Naomi's husband and his brother both have died. These young women are left widows with no means of support and with no hope for the future. They faced a terrible trial. It seemed that there only hope was to return to the home of their father and hope another man would eventually marry them.
    	\item Her Ambition \index[scripture]{Ruth!Ruth 01:06-07}(Ruth 1:6--7) -- Naomi has heard that the famine that caused her family to leave Bethlehem to begin with has ended and that there is bread in the land. So, she rises up to go home, and her two daughter-in-laws rise up to go with her.
    \end{compactenum}
    \item Ruth's \textbf{Challenge} \index[scripture]{Ruth!Ruth 01:08--15}(Ruth 1:8--15)
    \begin{compactenum}[A.][7]
    	\item An Expectation \index[scripture]{Ruth!Ruth 01:08-09}(Ruth 1:8--9) -- As they begin their journey, Naomi encourages both of these girls to return to the home of their mothers. She prays that the Lord will bless them, but she intends to send them away. Such is the condition of Naomi's heart that she would try to keep these women from going with her into the promised land of blessing.
    	\item An Explanation \index[scripture]{Ruth!Ruth 01:10-13}(Ruth 1:10--13) -- Naomi tries to persuade these women to go home because she has no more sons to marry the women and give them children under the law of the Levirate Marriage. And, even if she remarried and had children, these girls couldn't be expected to wait until these new sons were grown.
    	\item An Evacuation \index[scripture]{Ruth!Ruth 01:14}(Ruth 1:14) -- When Orpah heard this speech, she kissed her mother-in-law and turned around and went home.
    	\item An Examination \index[scripture]{Ruth!Ruth 01:14}(Ruth 1:14) -- After Orpah leaves, Ruth clings to Naomi. She is determined to stay with her mother-in-law. This allows us a glimpse into her heart. She shows us, by here actions that which is the best response in a time of challenge. The heart Ruth reveals to us is one of absolute devotion and commitment. You see, instead of driving us away from our commitments, the challenges of life should cause us to cleave to Him more strongly. In her commitment, Ruth demonstrates the fact that her heart had been changed and she was willing to follow a new Lord into a new land to live a new life.)
    \end{compactenum}
    \item Ruth's \textbf{Commitment} \index[scripture]{Ruth!Ruth 01:16--22}(Ruth 1:16--22)
    \begin{compactenum}[A.][7]
    	\item She Commits to a New Land \index[scripture]{Ruth!Ruth 01:16}(Ruth 1:16) -- She is willing to follow Naomi where ever she goes. She is willing to leave Moab behind forever and to follow Naomi to Israel.
    	\item She Commits to New Leadership \index[scripture]{Ruth!Ruth 01:16}(Ruth 1:16) -- She is willing to submit to Naomi and to allow Naomi to guide her life. (Ill. This is seen in the various times that Naomi gives Ruth advise concerning the manners and customs of Israel.)
    	\item She Commits to a New Lineage \index[scripture]{Ruth!Ruth 01:16}(Ruth 1:16) -- Ruth is willing to cut all ties with Moab. She wants to be a part of the nation which she has married into. She is ready to claim a new lineage.
    	\item She Commits to a New Lord \index[scripture]{Ruth!Ruth 01:16}(Ruth 1:16) -- This is perhaps the greatest statement Ruth makes. She is willing to give up the gods of Moab and follow the true and living God of Israel. This statement is her declaration of faith in Jehovah God.
    	\item She Commits with No Limits \index[scripture]{Ruth!Ruth 01:16}(Ruth 1:16) -- She tells Naomi that she is willing to commit to this new plan for life for as long as she lives. She even invokes the curse of God upon her life if she lets anything but death come between her and the commitment she has made.
    \end{compactenum}
\end{compactenum}
\textbf{Conclusion: }Ruth lived a consistent and steadfast life. She was brought into Israel, married an Israelite man named Boaz and became a part of the covenant people of the Lord. She became the great-grandmother to King David and an ancestor of the Lord Jesus Christ. All of this came to pass in her life because she was unwilling to change her mind, change her devotion or to change her direction. She kept on going in the face of adversity and she did not give up. Can the same be said about you this evening? Are you living a life that is steadfast and unmoveable, always abounding in the work of the Lord, 1 Cor. 15:58? Or has your devotion to the Lord tended to fluctuate with the changing tides of life? You know, you need God today, but you don't serve Him tomorrow. Where is your heart tonight? Are you more like Orpah who chose the easy path? Or, are you like Ruth who persevered through her difficulties to win the ultimate victory of faith? Are there issues that you need to get settled with the Lord this evening? If so, this altar is open for your use. Please use it as the Lord leads.
