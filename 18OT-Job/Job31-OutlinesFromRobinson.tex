\subsection{Outlines from Robinson}

\subsubsection{Job's Self-Vindication}
\textbf{Introduction:} Job concludes his speeches by a solemn, particular, and extended declaration of the purity and uprightness of his life. Especially reference to his private, as before to his public, conduct. Intended to silence his accusers and justify his complaints. Affords a  picture of an outwardly and blameless character.  A specimen, presented in beautiful language, of a pure morality accompanied with, and based upon, an ardent piety and genuine religion.\cite{robinson1876homiletical}:\footnote{Robinson, 1867}
\index[speaker]{Thomas Robinson!Job 31 (Job's Self-Vindication)}
\index[series]{Job (Thomas Robinson)!Job 31 (Job's Self-Vindication)}
\index[date]{2016/06/12!Job 31 (Job's Self-Vindication)}
\begin{compactenum}[I.][7]
    \item His Chastity \index[scripture]{Job!Job 33:01}(Job 31:1)
    \item His Honesty, Uprightness and Freedom from Covetousness (Job 31:5-8)
    \item His Freedom from Adulterous Desires and Practices  (Job 31:9-14)
    \item His Justice and Humanity to Servants or Slaves  (Job 31:13)
    \item His Benevolence and Kindness to the Poor  (Job 31:16,17)
    \item Denies all Vindictiveness in Reference to Enemies (Job 31:29,30)
    \item Job Declares his Humanity as a Householder (Job 31:31,32)
    \item Clears Himself from Secret and concealed Transgressions (Job 31:33,34)
    \item Job's final Desire and Challenge (Job 31:35--37)
    \item Job Finally Clears Himself of Injustice in his Business Transactions with his Fellow Man  (Job 31:38--40)
\end{compactenum}

